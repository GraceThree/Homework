\documentclass{article}
\usepackage{fancyhdr}
\usepackage{extramarks}
\usepackage{amsmath}
\usepackage{amsthm}
\usepackage{amssymb}
\usepackage{tikz}
\usepackage{microtype}

\author{Grace Unger}
\newcommand{\cmark}{\ding{51}}%
\newcommand{\xmark}{\ding{55}}%
\author{Grace Unger}
\title{Geometric Topology Homework}

\def\N{\mathbb{N}}
\def\Z{\mathbb{Z}}
\def\R{\mathbb{R}}
\def\Q{\mathbb{Q}}
\def\C{\mathbb{C}}
\def\K{\mathbb{K}}


\def\cA{\mathcal{A}}
\def\cB{\mathcal{B}}
\def\cC{\mathcal{C}}
\def\cF{\mathcal{F}}
\def\cH{\mathcal{H}}
\def\cL{\mathcal{L}}
\def\cN{\mathcal{N}}
\def\cO{\mathcal{O}}
\def\cP{\mathcal{P}}
\def\cR{\mathcal{R}}
\def\cS{\mathcal{S}}
\def\cI{\mathcal{I}}

\def\ee{\varepsilon}


%%%%%%%%%%%%%%%%%%%%%%
%Temperley-Lieb Diagram helper macros for Tikz
\tikzset{
  pt/.style={insert path={node[scale=2]{.}}},
  dnup/.style={insert path={ [pt] .. controls +(0,1) and +(0,-1) .. +(#1,2) [pt]}},
  dndn/.style={insert path={ [pt] .. controls +(0,1) and +(0,1) .. +(#1,0) [pt]}},
  upup/.style={insert path={ [pt] .. controls +(0,-1) and +(0,-1) .. +(#1,0) [pt]}},
}

%%%%%%%%%%%%%%%%%%%%%%%%%%%%%%%%%%%%%%%%%%%%

\begin{document}
\maketitle

\begin{enumerate}
    \item Let $f^{(n)}$ be the $n$th magic knitting. Show that $f^{(n)}-1$ is an element of the subalgebra of $TL_n(A)$ generated by the $e_i.$ That is, show that the idempotent really does behave like a projection.
    \begin{proof}
        First, remember that as an algebra, $TL_n(A)$ is generated by $1$ and the $e_i$. Further, $f^{(n)}1 = f^{(n)} \neq 0$ for $n\ge 1,$ but $f^{(n)}(f^{(n)}-1) = f^{(n)}-f^{(n)} = 0.$ This means that $f^{(n)}-1\in \ker f^{(n)}.$ Since $1$ and the $e_i$ span $TL_n(A)$, we may conclude that $f^{(n)}-1$ is in the span of the $e_i,$ since $1$ is not in the kernel of $f^{(n)}.$ Therefore, $f^{(n)}-1$ is in the subalebra generated by the $e_i.$
      \end{proof}
    \item Consider the $\pi$ rotations of the square about the cardinal axis and the axis perpendicular to the plane. Show that for $n\ge 3,$ these induce nonidentity involutions of $TL_n$ which fix $f^{(n)}.$
    \begin{proof}
      First, it is immediately clear that these are nonidentity involutions. Indeed, for $n\ge 3$, we can always pick some element of $TL_n$ which is unfixed by that rotation.
      
      Now, consider $\pi f^{(n)}.$ Since $TL_n$ is associative, $\pi f^{(n)}e_i = \pi(f^{(n)}e_i) = \pi 0 = 0.$ By problem 1, we also have that $\pi f^{(n)}-1$ is in the subalgebra generated by the $e_i.$ Since there is only one such $f^{(n)}$ satisfying these requirements, we must have $\pi f^{(n)} = f^{(n)}.$
    \end{proof}

    \item Prove that the Chebyshev polynomials satisfy $S_mS_n = \sum_{r} S_r$ for a particular value of $r$, given $m$ and $n.$
    \begin{proof}
      We recall that the Chebyshev polynomials are given by the relation $S_{n+1}(x) = xS_n(x) -S_{n-1}(x) $ subject to \[S_0(x) = 1,\, S_1(x) = x.\]
      Now, to find $r$, we take a few samples:
      \begin{align*}
        S_2(x) = xS_1(x)-S_0(x) & = x^2-1\\
        S_3(x) = xS_2(x)-S_1(x) &= x^3-2x\\
        S_4(x) = xS_3(x) -S_2(x) &= x^4-3x^2+1\\
        S_5(x) = xS_4(x) - S_3(x) &= x^5-4x^3+3x\\
        S_6(x) = x(S_5)(x) - S_4(x) &= x^6-5x^4+6x^2-1\\
        S_1S_0 &= x\\
        &= S_1\\
        S_1S_2 &= x^3-x\\
        &= S_3+S_1\\
        S_2S_3 &= x^5-3x^3+2x\\
        &= S_5 + S_3  + S_1\\
        S_1S_3 &= x^4-2x^2\\
        &= S_4 + S_2\\
        S_2S_4 &= x^6-4x^4+4x^2-1\\
        &=S_6 + S_4 + S_2.
      \end{align*}
      This gives us a very clear motivating pattern, and suggests that \[S_mS_n = \sum_{\substack{a=1\\a \text{ odd}}}^{m+n} S_a\] if $m+n$ is odd, or \[S_mS_n = \sum_{\substack{a=2\\a \text{ even}}}^{m+n} S_a\] if $m+n$ is even.
      Now, let $N$ be odd and suppose by way of induction that the result holds when $3<m+n<N$. Then let $m+n = N.$ By hypothesis, we know that \[S_{m-2}S_n = \displaystyle\sum_{\substack{a=1\\a \text{ odd}}}^{N-2}S_a.\] By the recurrence relation, $S_{m-2} = S_{m}-xS_{m-1},$ so we can rewrite this as \[(S_m -xS_{m-1})S_n = \sum_{\substack{a=1\\a \text{ odd}}}^{N-2}S_a\]
      or otherwise \[S_mS_n -x\sum_{\substack{a=2\\a \text{ even}}}^{N-1} S_a = \sum_{\substack{a=1\\a \text{ odd}}}^{N-2}S_a.\] Rewriting and again using the recurrence relation repeated, we get:
      \begin{align*}
        S_mS_n &= x\sum_{\substack{a=2\\a \text{ even}}}^{N-1} S_a - \sum_{\substack{a=1\\a \text{ odd}}}^{N-2}S_a\\
        &=\sum_{\substack{a=2\\ a \text{ even}}}^{N} xS_a-S_{a-1}\\
        &= \sum_{\substack{a=2\\ a \text{ even}}}^{N} S_{a+1}\\
        &= \sum_{\substack{a=1\\ a\text{ odd}}}^{N}S_a
      \end{align*} 
      exactly as needed. The proof for $m+n$ even is essentially identical, so we are done.
    \end{proof}
    \item The operation of placing a square containing a generating diagram of $TL_n$ and joining the points on the left and right without any crossings induces a linear map $\operatorname*{tr}:TL_n\to \C$. Show that $\operatorname*{tr}(xy) = \operatorname*{tr}(yx)$ and that $(x,y)\mapsto \operatorname*{tr}(xy)$ defines a bilinear form on $TL_n$. Show that if $A$ is not a root of unity, then this form is nondegenerate. 
    \begin{proof}
      Since for $D\in TL_n(A)$, the braid closure of $D$ has no crossings, we have that $\operatorname*{tr}(D)$ for a single diagram component is given by $(-A^{-2}-A^2)^m$ where $m$ is the number of closed curves in the closure of $D$. Extending this linearly, we have that $\operatorname*{tr}(D) = \sum_{d}(-A^{-2}-A^2)^{m(d)}$ where $d$ runs over all of the formal summands of $D$ and $m(d)$ is the number of closed curves in the closure of the diagram. \smallskip

      This means that our goal is to show that the number of closed curves of the braid closures of $xy$ and $yx$ are the same. To see this, notice that if $\gamma$ is a closed curve in $xy,$ then it meets internal and external edges of $x$ in some finitely many points, and similarly for $y.$ If the external edges are identified, then $\gamma$ meets the internal edges of $x$ and $y$, and the external edges of $x$ and $y$, at the same points. If the internal and external edges are swapped, as is done by taking $yx$, we will still be able to define a new curve $\gamma'$ which goes through the same edges, since some internal edge will still be connected along $\gamma'$ to the same external edge as it was connected to by $\gamma.$ \smallskip

      This means that interchanging $x$ and $y$ does not change the number of closed curves of the diagram, and so it does not change the trace. It remains to show that the trace is nondegenerate when $A$ is not a root of unity.\smallskip

      This is equivalent to showing that the Kauffman bracket is nonvanishing. But, as Lickorish notes, every diagram is a multiple of the empty diagram, and none of the terms can vanish if $A$ is not a root of unity. Therefore the Kauffman bracket is nonvanishing as well, so the form is nondegenerate. 
    \end{proof}
    \item Prove that the signature of a linking matrix of a framed link is not changed by K2 moves.
    \begin{proof}
      Suppose that we are doing a K2 move by blowing up the $i$th component around the $j$th component of a link. This affects the linking matrix by adding the $j$th column and row to the $i$th column and row, respectively.\footnote{From Dr Tubbenhauer's notes https://www.dtubbenhauer.com/slides/geotop2022/20-linking-matrix.pdf} Let $S_{pq}$ be the matrix so that $s_{pq} = 1$ when $(p,q) =(i,j)$ or $i=j$ and $s_{pq} = 0$ otherwise. We note that $S^*_{pq}$ is similar but with $s_{ji} =1$ instead. Then if $L$ is the linking matrix, $LS_{ij}$ is the matrix $L$ but with the $j$th row added to the $i$th row. Similarly, $S_{ij}^*L$ is the same matrix with the $j$ column added to the $i$th row. Together, this means that $S_{ij}^*LS_{ij}$ is the new linking matrix. Since $S_{ij}$ is clearly nonsingular, by Sylvester's law of intertia, $S_{ij}^*LS_{ij}$ and $L$ have the same signature. In other words, the K2 move preserves signature. 
    \end{proof}
    \item Suppose that $D$ is the 3 crossing diagram of the trefoil and let $M$ be the manifold defined by this surgery diagram with coefficient $1.$ Calculate $\cI_A(M)$ with $A = e^{\pi i/10}.$
      \begin{proof}
      We calculate this as $\cI_A(M) = \langle\mu\omega\rangle_{3_1}\langle\mu\omega\rangle_{U-}^\sigma\mu.$ The trefoil has framing $\pm 1$ depending on chirality. Here, we take a right-handed trefoil with framing $1$, so the linking matrix is the identity which has signature $1.$\smallskip

      Since $A^{20} = 1,$ by lemma 14.3 and linearity \[\langle \mu\omega\rangle = \dfrac{-\mu\overline G}{2A^{-28}(A^2-A^{-2})}\]
      with $G = \sum_{n=1}^{20}A^{n^2}.$ For $r=5$, we also have that \[\omega = \sum_{n=0}^{r-2}\Delta_nS_n(\alpha) =\Delta_0 + \Delta_1\alpha + \Delta_2(\alpha^2-1) + \Delta_3(\alpha^3-2\alpha)\]
      or rearranging terms:
      \[\omega = \Delta_0\alpha^0 + (\Delta_1-2\Delta_3)\alpha^1 + \Delta_2\alpha^2+\Delta_3\alpha^3.\]
      We then calculate $\langle\mu\omega\rangle_{3_1}$ linearly. First, $\langle \mu\Delta_0\alpha^0\rangle = 0$ as the empty diagram. Similarly, $\mu(\Delta_1-2\Delta_3)\langle\alpha\rangle = \mu(\Delta_1-2\Delta_3)\langle 3_1\rangle$ the Kaufman bracket of $3_1.$ This is known to be given by $\langle 3_1 \rangle = -A^5-A^{-3}-A^{-7}.$
      \end{proof}
    \item Show that \[(R\otimes I_4)(I\otimes R)(R\otimes I_4) = (I_4\otimes R)(R\otimes I_4)(I_4\otimes R)\] where \[R = (I\otimes F_d)C_{X,d}^2(I\otimes F^*_d).\]
      Also, show that $R$ is diagonal. 
      \begin{proof}
        That $R$ is diagonal is immediate - since $I$ and $F_d$ are diagonal, so is their tensor product. Also that must be shown is that $F_dX_d$ is diagonal. But $X_d$ is circulant by construction, and so the discrete fourier transform $F_d$ automatically diagonalizes it. This means that $(I\otimes F_d)C_{X,d}$ is also diagonal because is it block diagonal with nonzero entries consisting only of diagonal matrices, and so $R$ will be as well. Clearly, swapping the two qubits doesn't change this.\smallskip

        From David's paper, we know that we can write $R$ specifically as \[R = \begin{pmatrix}
          I & 0 & 0 & \dots & 0\\
          0 & F_dX_dF_d^* & 0 & \dots & 0\\
          0 & 0 & I & \dots & 0\\
          \vdots & \vdots &\vdots & \ddots & \vdots\\
          0 & 0 & 0 & \dots & I
        \end{pmatrix}\]
        Then when we expand, we get:
        \begin{align*}
          (R\otimes I_4) &= \begin{pmatrix} 
            I\otimes I_4 & 0 & 0 & \dots & 0\\
            0 & F_dX_dF_d^*\otimes I_4 & 0 &\dots & 0\\
            0 & 0 & I\otimes I_4 & \dots & 0\\
            \vdots & \vdots &\vdots &\ddots & \vdots\\
            0 & 0 & 0 & \dots & I\otimes I_4
          \end{pmatrix}\\
          (I_4\otimes R) &= \begin{pmatrix}
            R & 0 & 0 & 0\\0 & R & 0 & 0\\
            0 & 0 & R & 0\\0 & 0 &0  &R
          \end{pmatrix}
        \end{align*}
        Now, we have that $\operatorname*{rank}R = d^2$. Assuming that $d\ge2,$ let $T$ be $d^2\otimes d^2$ diagonal submatrix of $R\otimes I_4$ so that \[(I_4\otimes R)(R\otimes I_4) = \begin{pmatrix}
          RT & 0 & 0 & 0\\
          0 & R & 0 & 0\\
          0 & 0 & R & 0\\
          0 & 0 & 0 & R
        \end{pmatrix}\]
        where the lower diagonal blocks are just $R$ because all of the diaganal elements after the $2d$th term is $1.$ Explicitly then we get
        \begin{align*}
          T &= \begin{pmatrix}
            I\otimes I_4 & 0 & 0 & \dots & 0\\
            0 & F_dX_dF_x^*\otimes I_4 & 0 &\dots & 0\\
            0 & 0  & 1 & \dots & 0\\
            \vdots & \vdots & \vdots & \ddots & 0\\
            0 & 0 & 0 &\dots & 1
          \end{pmatrix}
        \end{align*}
        Now, since we took $d\ge 2,$ note that the first $4d$ diagonal terms of this matrix are $1$. But after the $2d+1$st diagonal term of $R,$ all the terms are $1.$ This means that we can write 
        \[RT = \begin{pmatrix}
          I_d & 0 & 0 &0 & 0 & 0 & \dots & 0\\
          0 & F_dX_dF_x^* & 0 &0 & 0 & 0 & \dots & 0\\
          0 & 0 & I_d&0 & 0 & 0 & \dots & 0\\
          0 & 0 & 0 &I_d & 0 & 0 & \dots & 0\\
          0 & 0 & 0 &0 & F_dX_dF_x^*\otimes I_4 & 0 & \dots & 0\\
          0 & 0 & 0 &0 & 0 & 1 & \dots & 0\\
          \vdots & \vdots & \vdots & \vdots &\vdots & \ddots &\vdots\\
          0 & 0 & 0 & 0  & 0 & \dots & 1
        \end{pmatrix}\] 
        Thankfully, the lower $3d^2$ terms of the tensor products are preseved throughout, so we only need to pay attention to $RT$. Now, we can compute the full products, taking advantage of diagonal matrices commuting:
        \begin{align*}
          (I_4\otimes R)^2(R\otimes I_4) &= \begin{pmatrix}
            R^2T & 0 & 0 &0\\
            0 & R^2 & 0 & 0\\
            0 & 0 & R^2 & 0\\
            0 & 0 & 0 & R^2
          \end{pmatrix}\\
          (R\otimes I_4)^2(I_4\otimes R) &= \begin{pmatrix}
            RT & 0 & 0 & 0\\
            0 & R & 0 & 0\\
            0 & 0 & R & 0\\
            0 & 0 & 0 & R
          \end{pmatrix}\begin{pmatrix} 
            I\otimes I_4 & 0 & 0 & \dots & 0\\
            0 & F_dX_dF_d^*\otimes I_4 & 0 &\dots & 0\\
            0 & 0 & I\otimes I_4 & \dots & 0\\
            \vdots & \vdots &\vdots &\ddots & \vdots\\
            0 & 0 & 0 & \dots & I\otimes I_4
          \end{pmatrix}
        \end{align*}
        As mentioned, we only need to focus on the first $d^2$ terms here, and using the construction of $RT$ we have above, this becomes:
        \begin{align*}
          &\,\begin{pmatrix}
            I_d & 0 & 0 &0 & 0 & 0 & \dots & 0\\
            0 & F_dX_dF_x^* & 0 &0 & 0 & 0 & \dots & 0\\
            0 & 0 & I_d&0 & 0 & 0 & \dots & 0\\
            0 & 0 & 0 &I_d & 0 & 0 & \dots & 0\\
            0 & 0 & 0 &0 & (F_dX_dF_x^*\otimes I_4)^2 & 0 & \dots & 0\\
            0 & 0 & 0 &0 & 0 & 1 & \dots & 0\\
            \vdots & \vdots & \vdots & \vdots &\vdots & \ddots &\vdots\\
            0 & 0 & 0 & 0  & 0 & \dots & 1
          \end{pmatrix}\\
          &=\begin{pmatrix}
            I_d & 0 & 0 &0 & 0 & 0 & \dots & 0\\
            0 & F_dX_dF_x^* & 0 &0 & 0 & 0 & \dots & 0\\
            0 & 0 & I_d&0 & 0 & 0 & \dots & 0\\
            0 & 0 & 0 &I_d & 0 & 0 & \dots & 0\\
            0 & 0 & 0 &0 & (F_dX_dF_x^*)^2\otimes I_4 & 0 & \dots & 0\\
            0 & 0 & 0 &0 & 0 & 1 & \dots & 0\\
            \vdots & \vdots & \vdots & \vdots &\vdots & \ddots &\vdots\\
            0 & 0 & 0 & 0  & 0 & \dots & 1
          \end{pmatrix}\\
          &= R^2T.
        \end{align*}
        Therefore, \[(R\otimes I_4)^2(I_4\otimes R) = \begin{pmatrix}
          R^2T & 0 & 0 &0\\
          0 & R^2 & 0 & 0\\
          0 & 0 & R^2 & 0\\
          0 & 0 & 0 & R^2
        \end{pmatrix}\]
        so this does indeed satisfy the Yang-Baxter equations.
      \end{proof}
      \item Show that $R$ satisfies the algebraic YBEs exactly when $RP$ it satisfies the brain YBEs.
      \begin{proof}
        For convenience we recall the algebraic YBEs:
        \begin{align*}
          R_{12} &= (R\otimes I)\\
          R_{13} & = (I\otimes P)(R\otimes I)(I\otimes P)\\
          R_{23} &= (I\otimes R)\\
          R_{12}R_{13}R_{23} &= R_{23}R_{13}R_{12}.
        \end{align*}
        and the braid YBE: \[(RP\otimes I)(I\otimes RP)(RP\otimes I) = (I\otimes RP)(RP\otimes I)(I\otimes RP).\]
        Suppose that $RP$ satisfies the braid $YBE.$ Then, since $R$ and $P$ have the same shape, we can use the identity $(A\otimes B)(C\otimes D) = (AC)\otimes (BD)$ and the corrolary $(A\otimes I)(I\otimes B) = (A\otimes B)$ to simplify this:
        \begin{align*}
          (RP\otimes I)(I\otimes RP)(RP\otimes I) &= (R\otimes I)(P\otimes I)(I\otimes R)(I\otimes P)(R\otimes I)(P\otimes I)\\
          (I\otimes RP)(RP\otimes I)(I\otimes RP) &=(I\otimes R)(I\otimes P)(R\otimes I)(P\otimes I)(I\otimes R)(I\otimes P)\\
        \end{align*}
        $(RP\otimes RP)(I\otimes RP) =  (RP\otimes (RP^2))$, so $(RP\otimes (RP)^2) = ((RP)^2 \otimes RP).$
      \end{proof} 
\end{enumerate}


\end{document}